\documentclass[a4paper,11pt]{article}

\usepackage[margin=1in]{geometry}
\usepackage[T1]{fontenc}
\usepackage[utf8]{inputenc}
\usepackage{newtxtext}
\usepackage[hidelinks]{hyperref}
\usepackage{enumitem}
\usepackage{titlesec}
\usepackage{multicol}
\pagenumbering{gobble}
\usepackage{tabularx}
\raggedbottom

% Section formatting
\titleformat{\section}{\bfseries\large}{}{0em}{}
\titleformat{\subsection}{\bfseries}{}{0em}{}
\titlespacing*{\section}{0pt}{1em}{0.3em}
\titlespacing*{\subsection}{0pt}{0.5em}{0.1em}
\setlist[itemize]{left=1.5em, label=--, noitemsep, topsep=0pt}

\begin{document}
	
	%====================
	% HEADER
	%====================
	\begin{center}
		{\Huge \textbf{Daniel Imanuel Manafe}}\\[0.5em]
		{\small
			\href{mailto:danielimanuelmanafe@mail.ugm.ac.id}{danielimanuelmanafe@mail.ugm.ac.id} \quad | \quad
			\href{https://wa.me/628811552351}{0881-1552-351} \quad | \quad
			\href{https://bit.ly/Niell_LinkedIn}{LinkedIn} \quad | \quad
			\href{https://bit.ly/Niell_3DModels}{3D Models}
		}
	\end{center}	
	%====================
	% PROFIL SINGKAT
	%====================
	\vspace{1em}
	Mahasiswa Universitas Gadjah Mada dengan fokus pada sistem instrumentasi dan kontrol. Memiliki pengalaman magang selama 1.5 tahun di berbagai sektor industri, termasuk makanan, jaringan, otomotif, dan pendidikan. Mampu bekerja tim, adaptif, serta memiliki kompetensi teknis dalam desain elektronik dan pengembangan sistem kendali.
	
	%====================
	% PENGALAMAN KERJA
	%====================	
	\vspace{1em}
	\begin{center}
		{\large\bfseries Pengalaman Kerja}\\[-0.6em]
		\noindent\makebox[\linewidth]{\rule{\textwidth}{0.4pt}}
	\end{center}

	\subsection*{GAMAFORCE – Gadjah Mada Flying Object Research Center \hfill \textnormal{\textit{Des 2024 – Sekarang}}}
	\textit{Head of Khageswara Division}
	\begin{itemize}
		\item Memimpin pengembangan sistem elektronik dan kontrol pesawat \textit{VTOL Tailsitter}.
		\item Mengatur integrasi mekatronika dan jadwal teknis tim divisi.
	\end{itemize}
	\noindent Bertanggung jawab atas arah teknis dan pencapaian performa UAV dalam kompetisi.\\

	\subsection*{OTS Merah Putih - LKFT UGM \hfill \textnormal{\textit{Sep 2024 – Sekarang}}}
	\textit{HMI Engineer}
	\begin{itemize}
		\item Membuat HMI OTS untuk proses produksi Asam Phosfat (PT. Petrokimia Gresik Production 3B).
		\item Membuat HMI OTS untuk proses produksi Ammonia (PT. Pupuk Kujang Production 3A).
	\end{itemize}
	\noindent Berperan sebagai pemimpin proyek digitalisasi sistem pelatihan operator.\\
	
	\subsection*{GAMAFORCE – Gadjah Mada Flying Object Research Center \hfill \textnormal{\textit{Des 2023 – Des 2024}}}
	\textit{Electronic Engineer}
	\begin{itemize}
		\item Wiring komponen elektronik untuk Fly-wing dan Fixed-wing.
		\item Merancang konsep mekatronika untuk mekanisme sayap lipat.
	\end{itemize}
	\noindent Mendukung pengembangan teknologi pesawat melalui desain dan integrasi elektronik.\\
	
	\subsection*{Robotik Academy \hfill \textnormal{\textit{Agu 2023 – Jan 2024}}}
	\textit{Student Trainee dan Desainer Sosial Media}
	\begin{itemize}
		\item Mengelola Instagram resmi dan mengajar Scratch dan Tinkercad.
	\end{itemize}
	\noindent Menggabungkan kreativitas digital dengan pengajaran dasar pemrograman anak.\\
	
	\subsection*{ExcellencIA Learning Center \hfill \textnormal{\textit{Nov 2023 – Jan 2024}}}
	\textit{Freelancer Video Editor}
	\begin{itemize}
		\item Mengedit video menggunakan Capcut secara remote.
	\end{itemize}
	\noindent Bekerja secara mandiri menghasilkan konten video edukatif berkualitas.\\
	
	\subsection*{Monitoring Watch Cardiovascular – PKM KC \hfill \textnormal{\textit{Feb 2023 – Des 2023}}}
	\textit{Electronic Engineer}
	\begin{itemize}
		\item Desain PCB dengan KiCAD 8.0 dan pengujian sensor detak jantung.
	\end{itemize}
	\noindent Merancang prototipe wearable kesehatan untuk penderita kardiovaskular.\\
	
	\subsection*{Edgytech Flathouse – PKM VGK \hfill \textnormal{\textit{Feb 2023 – Des 2023}}}
	\textit{Researcher}
	\begin{itemize}
		\item Merancang sistem monitoring energi terbarukan untuk rumah susun.
	\end{itemize}
	\noindent Meneliti pemanfaatan energi alternatif pada hunian vertikal ramah lingkungan.\\
	
	\subsection*{PT Asatama Teknologi Terpadu \hfill \textnormal{\textit{Okt 2021 – Des 2021}}}
	\textit{Network Engineer}
	\begin{itemize}
		\item Troubleshooting perangkat, topologi jaringan, instalasi dan konfigurasi Mikrotik.
	\end{itemize}
	\noindent Mendukung operasional jaringan dan perangkat di bidang layanan IT dan komunikasi.\\
	
	\subsection*{PT Prakarsa Alam Segar \hfill \textnormal{\textit{Jul 2021 – Okt 2021}}}
	\textit{Production Helper}
	\begin{itemize}
		\item Setup dan pengecekan wiring panel dan mesin produksi.
	\end{itemize}
	\noindent Membantu memastikan kesiapan mesin produksi mie instan di jalur utama.\\
	
	\subsection*{PT Dharma Controlcable Indonesia \hfill \textnormal{\textit{Jul 2019 – Agu 2019}}}
	\textit{Quality Control Helper}
	\begin{itemize}
		\item Sortir, inspeksi AQL, dan perbaikan barang NG.
	\end{itemize}
	\noindent Terlibat dalam pengendalian mutu komponen kabel otomotif selama magang.\\
	
	
	%====================
	% PENDIDIKAN
	%====================
	\vspace{1em}
	\begin{center}
		{\large\bfseries Riwayat Pendidikan}\\[-0.6em]
		\noindent\makebox[\linewidth]{\rule{\textwidth}{0.4pt}}
	\end{center}
	
	\subsection{Universitas Gadjah Mada \hfill \textnormal{\textit{Jul 2022 – Sekarang}}}
	Teknolgi Rekayasa Instrumentasi dan Kontrol \\
	GPA: 3.75 / 4.00
	
	\subsection{SMKN 1 Tambelang, Bekasitle \hfill \textnormal{\textit{Jul 2018 – Jun 2021}}}
	Jurusan Teknik Komputer dan Jaringan \\
	GPA: 82.6 / 100.0
	\begin{itemize}
		\item Magang di PT Dharma ControlCable Indonesia
	\end{itemize}
	
	%====================
	% PENGHARGAAN
	%====================
	\vspace{1em}
	\begin{center}
		{\large\bfseries Penghargaan}\\[-0.6em]
		\noindent\makebox[\linewidth]{\rule{\textwidth}{0.4pt}}
	\end{center}

	\begin{itemize}
		\item Juara 1 Divisi \textit{Technology Development} – KRTI (Kontes Robot Terbang Indonesia)
		\item Juara 2 Astranauts 2024 – PT Astra Digital Internasional
		\item Juara 1 SoTech (Social Technology Social \& Technology Innovation for Community Challenge) 2024 – PT Pertamina Patra Niaga
		\item Finalis PIMNAS (PKM-KC \& PKM-VGK) – KEMENDIKBUDRISTEK
		\item Awardee Astra InnovLab Batch 3
		\item Sertifikat Dicoding: Learn DevOps Basics
	\end{itemize}
	
	\pagebreak
	
	\vspace{1em}
	\begin{center}
		{\large\bfseries Keahlian}\\[-0.6em]
		\noindent\makebox[\linewidth]{\rule{\textwidth}{0.4pt}}
	\end{center}
	
	%====================
	% HARD SKILLS
	%====================
	\section*{Hard Skills}
	\begin{multicols}{2}
		\begin{itemize}
			\item C/C++, Python, MATLAB
			\item Arduino, ESP32, STM32
			\item SolidWorks, Inventor, Fusion 360
			\item KiCAD, Proteus, Eagle
			\item CX Programmer, CX Designer
			\item Cisco Packet Tracer, Mikrotik
			\item Aveva Intouch
			\item RDWorks lasercut
			\item Labview
			\item OriginLab
			\item Capcut, Figma, Illustrator
			\item Microsoft Office Suite
		\end{itemize}
	\end{multicols}
	
	%====================
	% SOFT SKILLS
	%====================
	\section*{Soft Skills}
	\begin{multicols}{2}
		\begin{itemize}
			\item Kepemimpinan
			\item Manajemen Waktu
			\item Problem Solving
			\item Adaptasi Cepat
			\item Komunikasi Efektif
			\item Kolaborasi Tim
			\item Pemikiran Kritis
			\item Kreativitas
		\end{itemize}
	\end{multicols}
	
	%====================
	% ORGANISASI & PANITIA
	%====================
	\medskip
	\begin{center}
		{\large\bfseries Organisasi dan Kepanitiaan}\\[-0.6em]
		\noindent\makebox[\linewidth]{\rule{\textwidth}{0.4pt}}
	\end{center}
	
	\subsection{\textnormal{Koordinator Media – PMK Sekolah Vokasi \hfill Februari 2023 – Februari 2024}}	
	\subsection{\textnormal{Media Staff – UKK (Unit Kerohanian Kristen) \hfill Februari 2023 – Februari 2024}}	
	\subsection{\textnormal{Koordinator Media – Dialog Lintas Agama UGM \hfill Agustus 2023 – Oktober 2023}}	
	\subsection{\textnormal{IT Staff – Gelanggang Expo \& UKK CUP \hfill Mei 2023 – Agustus 2023}}	
	\subsection{\textnormal{Humas \& IT – UKK CUP \hfill September 2022 – November 2023}}	
	
	%====================
	% KONTAK REKOMENDASI
	%====================
	\vspace{1em}
	\begin{center}
		{\large\bfseries Kontak Rekomendasi}\\[-0.6em]
		\noindent\makebox[\linewidth]{\rule{\textwidth}{0.4pt}}
	\end{center}
	
	\begin{center}
		\begin{multicols}{2}
			\textbf{Jans Hendry, S.T., M.Eng.} \\
			Dosen Pembimbing Akademik \\
			Email: \href{mailto:jans.hendry@ugm.ac.id }{jans.hendry@ugm.ac.id }
			
			\vfill\null
			\columnbreak
			
			\textbf{Ardhi Wicaksono Santoso., S.Kom., M.Cs.} \\
			Dosen Project LKFT UGM\\
			Email: \href{mailto:ardhi.wicaksono.s@mail.ugm.ac.id}{ardhi.wicaksono.s@mail.ugm.ac.id} 
		\end{multicols}
	\end{center}
\end{document}
